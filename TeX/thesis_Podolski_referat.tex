\documentclass[a4paper,12pt]{report}
\usepackage[utf8]{inputenc}

\usepackage{bsumain}
\usepackage{bsumasterelibtitle}

\usepackage{longtable}
\usepackage{csvsimple}
\usepackage{epstopdf}

\subfaculty{Кафедра математического моделирования и анализа данных}
\title{РАЗРАБОТКА АЛГОРИТМОВ И ПРОГРАММНОГО ОБЕСПЕЧЕНИЯ ТЕСТИРОВАНИЯ СЛУЧАЙНЫХ ПОСЛЕДОВАТЕЛЬНОСТЕЙ}
\author{Подольский Владислав Олегович}
\mentor{кандидат физико-математических наук, \\ доцент, М.С.Абрамович}

\begin{document}
  \maketitle

  \newpage
  \chapter*{Реферат}

  Магистерская диссертация, 53 страницы, 1 таблица, 15 источников, 2 приложения.

  ГЕНЕРАТОР СЛУЧАЙНЫХ ЧИСЕЛ, ЭНТРОПИЯ, ГЕНЕРАТОР ИСТИННО СЛУЧАЙНЫХ ЧИСЕЛ, ИСТОЧНИК СЛУЧАЙНОСТИ, ИСТОЧНИК ЭНТРОПИИ, ОЦЕНКА ЭНТРОПИИ, МИНИМАЛЬНАЯ ЭНТРОПИЯ.

  \textit{Объект исследования} --- выходные последовательности источников случайности.

  \textit{Цель работы} --- разработка алгоритмов и программного обеспечения оценки энтропии выходных последовательностей источников случайности. В работе рассматриваются подходы к определению количества неопределенности в последовательностях. Исследуются статистические свойства последовательностей и строятся оценки минимальной энтропии.

  \textit{Результат работы} --- описаны подходы и алгоритмы для оценки качества источников случайности, разработано универсальное программное обеспечение для этой оценки. Построены оценки минимальной энтропии для кольцевого осциллятора. Показана практическая применимость разработанного программного обеспечения.

  \textit{Методы исследования} --- математическая статистика, численные эксперименты. Область применения: разработанные алгоритмы и ПО могут применяться для тестирования источников случайности на соответствие требованиям различных стандартов, в том числе СТБ 34.101.27-2011 <<Требования безопасности к программным средствам криптографической защиты информации>>.

  \newpage
  \chapter*{Abstract}

  Master thesis, 53 pages, 1 table, 15 sources, 2 appendices.

  RANDOM NUMBER GENERATOR, ENTROPY, TRUE RANDOM NUMBER GENERATOR, RANDOMNESS SOURCE, ENTROPY SOURCE, ENTROPY ESTIMATION, MINIMAL ENTROPY.

  \textit{Object of research} --- output sequences of randomness sources.

  \textit{Purpose of research} --- development of algorithms and software for entropy estimation of the output sequences of randomness sources. The paper discusses approaches to determining the amount of uncertainty in the sequences. Statistical properties of sequences were studied and estimates of the minimal entropy were evaluated.

  \textit{Results of research} --- the thesis describes the approaches and algorithms for evaluating the quality of randomness sources, and universal software for this assessment was developed. Estimation of minimum entropy for a ring oscillator was built. The practical applicability of the developed software was shown.

  \textit{Research methods} --- mathematical statistics, numeric experiments. Applications: developed algorithms and software can be used to test the randomness sources for compliance with various standards, including STB 34.101.27-2011 << Safety requirements for software of cryptographic protection of information >>.

\end{document}
